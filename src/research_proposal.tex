
\documentclass[12pt]{article}

\usepackage{csvsimple}
\usepackage{hyperref}
\usepackage{tabularx}
\usepackage{geometry}

\hypersetup{
    colorlinks=true,
    linkcolor=blue,
    filecolor=magenta,      
    urlcolor=blue,
    pdftitle={Markus Amano Research Proposal}
    }

\geometry{left=15mm, right=15mm, top=15mm, bottom=15mm}

\begin{document}

\begin{center}
  \begin{tabularx}{\textwidth} { 
      >{\raggedright\arraybackslash}X 
    >{\raggedleft\arraybackslash}X  }
    \huge Markus A.G. Amano & Research Proposal\\
    \hline
    Job Number: \textbf{W22219} & \\
  \end{tabularx}
\end{center}

% Research proposal (description of your research objectives over the timescale of the position, and the laboratory names with which you wish to collaborate; 3 pages max)

% Research Summary 
% Nuclear structure and reactions reveal various aspects of quantum many-body systems due to the assembly and disassembly of protons and neutrons. 
% In the Nucleon Many-body Theory Laboratory, we aim to understand such dynamics of nuclei and to construct theoretical models to describe them. 
% Our research topics include nuclear structure issues such as deformation, shell structure, and clustering of unstable nuclei, and nuclear reactions in the Universe where elements originate. 
% In addition to this fundamental research, we are also developing nuclear reaction databases for applications in various scientific and technological fields such as medicine and industry.

\paragraph{Proposal Abstract}

% objectives
In all generality, my goal is to improve hadronic models better fit hadron spectra and to improve theoretical predictions of multi nucleon objects. 
%
My proposed method of research is with holographic models.
%
These models have the mysterious fifth dimension, but such a dimension is the geometric realization of the renormalization group (RG) flow of the theory from IR to the UV.
%
The leading holographic model is holographic Sakai-Witten-Sugimoto model that is a bottom-up model with type IIA string theory.
%
The strongly coupled hadron physics is said to be dual to weakly coupled supergravity theory.
%
The model uses topological soliton in the bulk to model baryons (skymions) in the boundary with confinement and broken chiral symmetry.
%
I propose to two topics of research: further understanding of the moduli space for the self-dual instatons/dual to skrymion.
%
I propose two research topics.
%
Extend the moduli analysis currently being done.
%
The instatons in flat space represent the transitions of QCD vaccua.
%
In the hologrpahic, negativly curved space case, the instatons are dual to skymions, so the moduli of such holographic solitons are dual to the baryons.
%
Because the holographic solitons are topological, within the Sakai-Witten-Sugimoto number of baryons is a topological invariant.
%
Further, analysis of the moduli spce can determine where it is a Riemannian manifold or not.
%
Similar to the flat space case, such understanding of the moduli can help construct multi baryon solutions quantized degress of freedom.
%
The other proposed line of research would involve modeling (high) spin hydronic spin states with a scalar theory.
%
Within holographic theories, one can introduce two-component complex scalars such that they have a broken chiral symmetry.
%
It is hypothesised that the broken chiral creates non-topological domains walls.
%
It was novel to test the effectiveness of such a model to model confinement and it's effectiveness to model low mass hadron states.

% laboratory and researcher names one wishes to work with
I believe that my research topic and work would most closely aglign with Dr. Masaaki Kimura and his group at the Nucleon Many-body Theory laboratory.
%
I also believe that is there is mutual befinit to our collaboration.
%
The majority of my work it more along the lines of finding QCD adjacent theories.
%
The main focus, not to compare with experimental data, but to expolore the strongly coupled field theory land-scape.
%
I believe that Dr. Kimura, his group, and I could perform more phenomological approach to model building with there
%
\large{\bf{ADD MORE HERE}}

% time scale
I would expect a year to find interestingly novel results.
%
After model building, I would expect find analytically or numberical results after one or a few months.

\paragraph{Proposed Method of Research}

For research pertaining to hadron modeling, first it is important for the research is to construct the model in the terms of a holographic Lagrangian.
%
That is a 5D hologrpahy.
%
Immidiete candidates would be variations of the Sakai-Witten-Sugimoto.
%
If the model has no known simple homogenous solutions of the equations of motion, I would first exploring finding such simple solutions.
%
Such solutions are usually thermaldynamically the most favorable solution.
%
Saying such thing, I would explore the thermalydyanmics of such a solution on the boundary.
%
Further, analysis can be done by calculating meson spectra and couplings if admited by the model.
%
Otherwise, for more non-trival, non-homogenous solutions, I would numberically solve such solutions as a boundary value problem (BVP).
%
Calculating the spectra and couplings of meson similarly requires one to solve a BVP.
%
\large{\bf{ADD MORE HERE}}

For research involving moduli space of curved space instatons involves more mathematical techqiques involving algebraic topology.
%
Essentially, the method of research here would be to concptually tackle the problem as a simple (but distinct problem first) or to exploit the symmetries of the problem.
%
\large{\bf{ADD MORE HERE}}

\paragraph{Expected Results}

\large{\bf{ADD MORE HERE}}

\paragraph{Future Work}

For the moduli space reaserch, I would further develop methods to construct more generic solutions, similar to flat space case.
%
Such solutions would be parametrized by the moduli.
%
Such moduli can be quantized with the \bf{TBD} method quantization.
%
\large{\bf{ADD MORE HERE}}

\end{document}
