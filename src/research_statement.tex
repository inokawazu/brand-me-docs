\documentclass[fontsize=12pt]{article}

\usepackage{csvsimple}
\usepackage{hyperref}
\usepackage{tabularx}
\usepackage{geometry}

\hypersetup{
    colorlinks=true,
    linkcolor=blue,
    filecolor=magenta,      
    urlcolor=blue,
    pdftitle={Markus Amano Research Statement}
    }

\geometry{left=15mm, right=15mm, top=15mm, bottom=15mm}

\begin{document}

\begin{center}
  \begin{tabularx}{\textwidth} { 
      >{\raggedright\arraybackslash}X 
    >{\raggedleft\arraybackslash}X  }
    \huge Markus A.G. Amano & Research Statement\\
    \hline
    Job Number: \textbf{W22219} & \\
  \end{tabularx}
\end{center}

% holography exerpt

\paragraph{Rotating Plasmas}

% background
Quark Gluon Plasmas (QGP) produced at Brookehaven's RHIC (Relativistic Heavy Ion Collider) and CERN have been of interested to both theoretical and experimental communities.
Heavy ions (like Au or Pb), collide to produce such a fluid.
Head-on collisions have negligible voricity.
Otherwise, collisions of center were found by the STAR collaboration to have the highest voricity in a natural phenomenon.
It's well known that the QGP thermlizes quickly with respect to the time scale of the system.
Despite the large couple hydrodynamical descriptions worked ``unaturally'' well.

% research
With the AdS/CFT holographic duality, the we calculated transport coefficients of an analogous strongly CFT to that of QCD at finite temperature and voricity.
The dual of a rotating CFT is a rotating blackhole in 5 dimensional AdS. This black hole is an Myers-Perry AdS blackhole.
Over the course of three papers, we find the three results. 
One, for the hydrodynamic regime, the measure of the effeciveness of hydrodynamical increases as extremality is approached.
Two, locally, the transport coefficients of dual plasma equivalent to the transport of boasted relativistic fluid.
Three, despite the lack of seperability out of time correlators (OTOC) were calculated along with the associated ``chaos'' transport - the lyapunov exponent and butterfly velocity.

\paragraph{Sakai-Witten-Sugimoto Instantons}

% background
% research

\paragraph{Two-Component Scalar Model}

% background
% research

\end{document}
