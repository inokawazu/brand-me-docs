\documentclass[fontsize=12pt]{article}

\usepackage{csvsimple}
\usepackage{hyperref}
\usepackage{tabularx}
\usepackage{geometry}
\usepackage{dsfont}

\hypersetup{
    colorlinks=true,
    linkcolor=blue,
    filecolor=magenta,      
    urlcolor=blue,
    pdftitle={Markus Amano Research Statement}
    }

\geometry{left=15mm, right=15mm, top=15mm, bottom=15mm}

\begin{document}

\begin{center}
  \begin{tabularx}{\textwidth} { 
      >{\raggedright\arraybackslash}X 
    >{\raggedleft\arraybackslash}X  }
    \huge Markus A.G. Amano & Research Statement\\
    \hline
    Job Number: \textbf{W22219} & \\
  \end{tabularx}
\end{center}

% holography exerpt

\paragraph{Rotating Plasmas}

% background
Quark Gluon Plasmas (QGP) produced at Brookehaven's RHIC (Relativistic Heavy Ion Collider) and CERN have been of interested to both theoretical and experimental communities.
Heavy ions (like Au or Pb), collide to produce such a fluid.
Head-on collisions have negligible voricity.
Otherwise, collisions of center were found by the STAR collaboration to have the highest voricity in a natural phenomenon.
It's well known that the QGP thermlizes quickly with respect to the time scale of the system.
% Despite the large couple hydrodynamical descriptions worked ``unaturally'' well.

% research
With the AdS/CFT holographic duality, the we calculated transport coefficients of an analogous strongly CFT to that of QCD at finite temperature and voricity.
The dual of a rotating CFT is a rotating blackhole in 5 dimensional AdS. 
This black hole is the so called five dimensional Myers-Perry AdS blackhole (5DMPAdS).
Over the course of three papers, we find the three results. 
One, for the hydrodynamic regime, the measure of the effeciveness of hydrodynamical increases as extremality is approached.
Two, locally, the transport coefficients of dual plasma equivalent to the transport of boasted relativistic fluid.
Three, despite the lack of seperability out of time correlators (OTOC) were calculated along with the associated ``chaos'' transport - the lyapunov exponent and butterfly velocity.

In my first paper of the subject, we caluclated hydrodynamic and non-hydrodynamic transport related quatities of a rotating blackhole the case of small and larger temperature.
We focused on the latter, large temperature case due to it's promise to qualatatively model hydrodynamic in rotating QGP.
For the smaller temperature, we focused on calculating non-hydrodynamic transport as Quasinormal modes
  \footnote{QNMs are dissipative when the temperature is not zero.}.
Furthermore about the gravitational background, foor 5D the number of independent planes of rotation is two, so the number of rotation parameters is two.
Using the simply spinning configuration of the parameters (where they are set equal to eachother), we use the known fact that the background geometry has enhanced symmetry ($U(1)\times U(1) \rightarrow U(2)$).
This enhanced symmetry allowed us to seperate the equations of motions of gravitational perturbations, such that we only had to solve ODEs instead of PDEs.
Starting with the ``small'' temperature black hole (with a horizon of spherical topology), we calculated QNMs.
On the dual theory, these QNMs correspond to expectation values of one point functions of the conformal stress energy temsor 4D theory.
The topology of the 4D theory is $\mathds{R}^1 \times S^3$, this topology is dual to a confining phase of the dual 4D theory.
The horizon itself is dual to a non-zero temperature in the field theory.
The rotation was scanned numberically analyze it effect on the dissipative and propagative properties of the 4D theory.
We confirmed a previous result for large non-extremal rotation, the boundary becomes spacelike
  \footnote{The boundary ``spins faster than light''.}, 
and the background suffers from a linear instability since some of the QNMs' frequencies' imaginary parts become positive.
We found discrimination between the direction the propagating gravitational modes which traveled with and against the direction of rotation.
For large but finite temperature, we found Quasinormal Frequencies (QNF) could be approximated by a hydrodynamic expansion, despite such an expansion being ill-defined in the small temperature case.

For the other paper of the we analyzed the large temperature limit where we scale the temperature and the holographic direction to be very large.
This limit is also known as the ``planar limit'' since the resulting geometry of the dual field theory is planar (ie $\mathds{R}^{1,3}$).
In this limit, we calculated QNMs and hydrodynamic transport coefficients for three sectors (which decouple in the large black hole limit).

\bf{TBD}

\paragraph{Sakai-Witten-Sugimoto Instantons}

% background
% research

\paragraph{Two-Component Scalar Model}

% background
% research

\end{document}
