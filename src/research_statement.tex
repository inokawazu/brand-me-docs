\documentclass[12pt]{article}

\usepackage{csvsimple}
\usepackage{hyperref}
\usepackage{tabularx}
\usepackage{geometry}
\usepackage{dsfont}

\hypersetup{
  colorlinks=true,
  linkcolor=blue,
  filecolor=magenta,      
  urlcolor=blue,
  pdftitle={Markus Amano Research Statement}
}

\geometry{left=15mm, right=15mm, top=15mm, bottom=15mm}

\begin{document}

\begin{center}
  \begin{tabularx}{\textwidth} { 
      >{\raggedright\arraybackslash}X 
    >{\raggedleft\arraybackslash}X  }
    \huge Markus A.G. Amano & Research Statement\\
    \hline
    Job Number: \textbf{W22219} & \\
  \end{tabularx}
\end{center}

\paragraph{Sakai-Witten-Sugimoto Instantons}
% background
With the holographic Quantum Chromodynamics (QCD) models such as Sakai-Witten-Sugimoto a promising method of doing tractable calculations on a strongly coupled field theory with broken chiral symmetry is available.
%
This field of modeling QCD with holography is simply noted as AdS/QCD where ``AdS'' means a spacetime that is negatively curved and locally, asymptotically AdS.
%
There are still questions to be answers nonetheless about holographic models of QCD.
%
The holographic models have the duality where bulk instanton as dual to skyrmions in the boundary theory (which represent the baryons of QCD).
%
The number of baryons is topologically protected and is mathematically the second Chern class of the Yang-Mills field.
%
In contrast to the flat duality, the the negative curvature naturally sets a size scale of bulk soliton/boundary skyrmion.
%
Because of the negative curvature however, the size moduli of to be found instantons are zero.
%
One can introduce a Chern-Simons to provide an repulsion to set the size of bulk solitons to a finite but fixed size.

The Sakai-Witten-Sugimoto model is an example of a top-down AdS/QCD model, derived from a string embedding of $D8-\overline{D8}$ branes in type IIA string theory.
%
To use supergravity in such models, the 't Hooft limit is take where the number of colors and 't Hooft coupling, $\lambda$, is taken to be large.
%
For large $\lambda$, size of the soliton can be said to be roughly inversely proportional to the size of the soliton.
%
Also the supergravity approximation so in practice the calculations are down with field theory in a curved background.
%
For large $\lambda$, soliton's size is small compared to the curvature scale of the bulk. 
%
So, for such a setup, it has been done where the soliton was approximated by flat space instanton.

% Research
This ongoing research is to calcualte the dimension of the moduli of such instantons (solitons) which are then dual to baryons (skyrmions) on the boundary.
%
We are motivated to understand the moduli of the dual skyrmions.
%
This research is similar to the seminal works of Atiyah and others with flat space instantons.
%
For flat space instantons, moduli is a Riemannian manifold, but this is not a priori the case for the negatively curved space.

Moduli are all the transformations that leave the solutions at the some energy, ie all the non-gauge transformations that leave the solution's energy unchanged.
%
These degrees of freedom are commonly known as moduli and corresponds to transformations of the instanton (translations, rotations, SU(2) rotations).
%
For $SU(2)$, instantons in $\mathds{R}^4$ flat space, the dimension of the moduli is $8k - 3$ where $k$ is the number of instantons and $-3$ are for setting the gauge.
%
This can be calculated with the Atiyah-Singer (AS theory).
%
Commonly, physicists use the heat kernel "trick" to calculated the index and then use that to calculate the dimension of the moduli space.

Nevertheless, we are interested in the AdS case. 
%
AdS-like bulks have a conformal, time-like boundary.
%
This boundary as effects on the number of moduli.
%
The original theoretic basis of the Atiyah-Singer theorem, only applied to manifolds without boundaries. 
%
Atiyah-Patodi-Singer index theorem (APS theorem) takes into account of the boundary and additional term that proportionally sums the number of boundary zero modes and the spectral asymmetry\
  \footnote{The $\eta$ invariant.}.
  %
We believe, for negatively curved compact space with a single boundary, the moduli will be reduced to $6k - 3$ where boundary takes away $k$ via boundary conditions and the negative curvature sets the scale to take away another $k$.
%
For more boundaries, the number of moduli taken away is proportional to the number of disjoint boundaries.

\paragraph{Rotating Plasmas}

% background
Quark Gluon Plasmas (QGP) produced at Brookehaven's RHIC (Relativistic Heavy Ion Collider) and CERN have been of interested to both theoretical and experimental communities.
%
Heavy ions like Au or Pb are collide to produce such a plasma.
%
Head-on collisions have negligible vorticity.
%
Otherwise, collisions of center were found by the STAR collaboration to have the highest vorticity in a natural phenomenon.
%
It's known that the QGP thermalizes quickly with respect to the time scale of the system.

% research
With the AdS/CFT holographic duality, the we calculated transport coefficients of an analogous strongly CFT to that of QCD at finite temperature and vorticity.
%
The dual of a rotating CFT is a rotating black hole in 5 dimensional AdS. 
%
This black hole is the so called five dimensional Myers-Perry AdS black hole (5DMPAdS).
%
Over the course of three papers, we find the following three main results. 
%
One, for the hydrodynamic regime, the measure of the effectiveness of hydrodynamical increases as extremality is approached.
%
Two, locally, the transport coefficients of dual plasma were found and it was found that it's linear hydrodynamic transport is equivalent to the causal, stable, linear hydrodynamic transport of a boasted relativistic fluid.
%
Three, despite the lack of separability out of time correlators (OTOC) were calculated along with the associated ``chaos'' transport - the Lyapunov exponent and butterfly velocity.
%

In my first paper of the subject, we calculated hydrodynamic and non-hydrodynamic transport related quantities of a rotating black hole the case of small and larger temperature.
%
We focused on the latter, large temperature case due to it's promise to qualitatively model hydrodynamic in rotating QGP.
%
For the smaller temperature, we focused on calculating non-hydrodynamic transport as Quasi-normal modes\footnote{\
  QNMs are dissipative when the temperature is not zero.
}.
%
Furthermore about the gravitational background, for 5D the number of independent planes of rotation is two, so the number of rotation parameters is two.
%
Using the simply spinning configuration of the parameters (where they are set equal to each other), we use the known fact that the background geometry has enhanced symmetry ($U(1)\times U(1) \rightarrow U(2)$).
%
This enhanced symmetry allowed us to separate the equations of motions of gravitational perturbations, such that we only had to solve ODEs instead of PDEs.
%
Starting with the ``small'' temperature black hole (with a horizon of spherical topology), we calculated QNMs.
%
On the dual theory, these QNMs correspond to expectation values of one point functions of the conformal stress energy tensor 4D theory.
%
The horizon itself is dual to a non-zero temperature in the field theory.
%
The rotation was scanned numerically analyze it effect on the dissipative and propagative properties of the 4D theory.
%
We confirmed a previous result for large non-extremal rotation, the boundary becomes space-like\footnote{\
  The boundary ``spins faster than light''.
}, and the background suffers from a linear instability since some of the QNMs' frequencies' imaginary parts become positive.
%
We found discrimination between the direction the propagating gravitational modes which traveled with and against the direction of rotation.
%
For large but finite temperature, we found Quasi-normal Frequencies (QNF) could be approximated by a hydrodynamic expansion, despite such an expansion being ill-defined in the small temperature case.

For the other part of the paper of the we analyzed the large temperature limit where we scale the temperature and the holographic direction to be very large.
%
This limit is also known as the ``planar limit'' since the resulting geometry of the dual field theory is planar (i.e. $\mathds{R}^{1,3}$). 
%
In this limit, we calculated QNMs and hydrodynamic transport coefficients for three sectors (which decouple in the large black hole limit).
%
We found that locally, the QNMs hydrodynamic sector was equivalent to a boast relativistic fluid as was already shown.
%
As was shown in the previous section, the direction of the rotation discriminates modes that propagate parallel or anti-parallel to the rotation.
%
The fluid was found to be as first order, hydrodynamically causal and and stable.


\end{document}
